\documentclass{article}
\usepackage{listings}
\usepackage{verbatim}
\usepackage{color}
\usepackage{xcolor}
\definecolor{dkgreen}{rgb}{0,0.6,0}
\definecolor{gray}{rgb}{0.5,0.5,0.5}
\definecolor{mauve}{rgb}{0.58,0,0.82}
\lstset{frame=tb, language=Java, aboveskip=3mm, belowskip=3mm, showstringspaces=false, columns=flexible, basicstyle = \ttfamily\small, numbers=none, numberstyle=\tiny\color{gray}, keywordstyle=\color{blue}, commentstyle=\color{dkgreen}, stringstyle=\color{mauve}, breaklines=true, breakatwhitespace=true, tabsize=3 }
\title{Project 4}
\author{Zhiyu Mei, Yuxin Qin, Boshi Tang}
\begin{document}
\maketitle

\section{Multiple threads}
Normally the server runs multiple threads at the same time. There are 3 kinds of threads implemented,

\subsection{Client thread}
In \texttt{class BlockChainMinerClient}, this kind of threads takes charge of building channels, talking to other servers, boardcasting and get data from other servers.

\subsection{Computing thread}
In \texttt{class Computer}, there are only 1 computing thread, run method \texttt{compute\_nonce()}.

\subsection{Server thread}
The server thread is the main thread. Validating and storing the blocks into memory, and output logs. At the same time, notify client threads and computing thread at proper time.

\section{Data Structure}
The server uses a \texttt{HashMap} to store all the blocks it retrived, using the hash string of the block as the keys. Moreover, it preserves block chain tree data structure, by storing 
all of the block chain tails in a priority queue, which sort the chain tails with the length of the chains (if the length is the same, sort with hash value.) 
The blocks form a chain, with PrevHash values as pointers pointing to the previous block of the chain. We can easily maintain the longest chain by calling 
\texttt{peek()} of the priority queue, and complete the chain along the pointers.

\section{Restore}
New blocks are stored after validation. If validation fails, either the blocks is not legit by hash, transactions, or the server can not find the previous block in his memory. 
When the latter situation occurs, the server should try to restore the chain by asking other servers about the previous blocks of this block. 
If the chain of this block can not be retrived completely, or forms a cycle, discard the block immidiately.

\section{Test Cases and Start Script}
A simple test case is presented in \texttt{test.sh}. It starts 3 servers, pushing 100 transactions to one of them.

Another more complex test case is that 1 server startup 100 seconds later than the other 2 servers, and tries to recover the blocks from others. The scripts are 
\texttt{test\_recover.sh} and \texttt{client\_start.sh}.

Another test case about balance is \texttt{test\_run.sh} provided by example.



\end{document}